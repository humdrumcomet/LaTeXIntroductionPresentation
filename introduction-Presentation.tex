% arara: lualatex: { shell: yes }
% arara: makeglossaries
% arara: biber
% arara: lualatex: { shell: yes }
% arara: lualatex: { shell: yes }


% Demo report and presentation prepared by Aaron English (humdrumcomet on github)
\documentclass{beamer}
% Beamer format setup
\usetheme[progressbar=frametitle, numbering=fraction]{metropolis}
\setbeameroption{show notes on second screen}
\setbeamertemplate{note page}[plain]
\setbeamerfont{note page}{size=\Large}
\setbeamercovered{transparent, still covered={\opaqueness<1->{65}},again covered={\opaqueness<1->{65}}}

\usepackage{fontspec}
\usepackage{geometry} % useful for defining page geometries
\usepackage[style=ieee,backend=biber]{biblatex} % for handling bibliographies
\usepackage{float} % added control over
\usepackage{graphicx} % tools for inclusion of graphics
\usepackage{booktabs} % adding commands to improve the look of tables
\usepackage{csvsimple} % simplify table creation by importing .csv files directly
\usepackage{siunitx} % consistent notation and correct formatting of units
\DeclareSIUnit{\torr}{Torr} % Custom unit definition
\usepackage{xcolor} % to access the named colour LightGray
\definecolor{LightGray}{gray}{0.9}
\usepackage{minted} % inclusion of code blocks with syntax highlighting and
\setminted{linenos=flase, autogobble, fontsize=\scriptsize, breaklines=true, bgcolor=LightGray} % set global options for minted environments
\setmintedinline{bgcolor={}}
\usepackage{chemformula} % for writing chemical formulae
\usepackage[useregional]{datetime2}
\usepackage[duration=30]{pdfpc}
\usepackage[debug, section=section, acronym, symbols]{glossaries} % Glossaries package


% siunitx package setup
\sisetup{
load-configurations = abbreviations,
detect-family = true,
detect-weight = true,
per-mode = reciprocal,
input-digits = { 0123456789\pi\dots },
input-comparators = { <=>\approx\ge\geq\gg\le\leq\ll\sim\gtrsim\lesssim },
table-figures-integer = 3,
table-figures-decimal = 2,
table-auto-round,
}%

% Bibliography .bib file locations should maybe be local?
\addbibresource[location=local]{./accelerators.bib}

% Add Constants list using glossary
\newglossary[cgls]{constants}{cstog}{cstig}{Constants}
% Alphabetize glossary and acronyms list
\makeglossaries

%%% Constants
%%%%%%%%%%%%%%%%%%%%%%%%%%%%%%%%%%%%%%%%%%%%%%%%%%%%%%%%%%%%%%%%%%%%%%%%%%%%%%%%%%%%%%%%
\newglossaryentry{c}
{
    type=constants,
    name={\ensuremath{c}},
    description={\mbox{} speed of light (\SI{2.997924e8}{\meter\per\second})}
}
\newglossaryentry{boltz}
{
    type=constants,
    name={\ensuremath{k_{B}}},
    description={\mbox{} Boltzmann constant (\SI{8.617330e-5}{\electronvolt\per\kelvin})}
}
\newglossaryentry{eps0}
{
    type=constants,
    name={\ensuremath{\varepsilon_{0}}},
    description={\mbox{} permittivity of free space (\SI{8.854187e-12}{\farad\per\meter})}
}
\newglossaryentry{eq}
{
    type=constants,
    name={\ensuremath{e}},
    description={\mbox{} elementary charge (\SI{1.602176e-19}{\coulomb})}
}

%%% Symbols
%%%%%%%%%%%%%%%%%%%%%%%%%%%%%%%%%%%%%%%%%%%%%%%%%%%%%%%%%%%%%%%%%%%%%%%%%%%%%%%%%%%%%%%%
\newglossaryentry{mg}
{
    type=symbols,
    name={\ensuremath{G}},
    description={Mosfet Gate}
}
\newglossaryentry{md}
{
    type=symbols,
    name={\ensuremath{D}},
    description={Mosfet Drain}
}
\newglossaryentry{ms}
{
    type=symbols,
    name={\ensuremath{S}},
    description={Mosfet Source}
}
\newglossaryentry{press}
{
    type=symbols,
    name={\ensuremath{P}},
    description={Pressure (units of \si{\torr})}
}
\newglossaryentry{time}
{
    type=symbols,
    name={\ensuremath{t}},
    description={time (units of minutes)}
}
\newglossaryentry{ener}
{
    type=symbols,
    name={\ensuremath{E}},
    description={Energy}
}
\newglossaryentry{efermi}
{
    type=symbols,
    name={\ensuremath{E_{f}}},
    description={Fermi Energy of a Material}
}
\newglossaryentry{fermiDist}
{
    type=symbols,
    name={\ensuremath{f(E)}},
    description={The Fermi-Dirac distribution}
}
\newglossaryentry{econd}
{
    type=symbols,
    name={\ensuremath{E_{C}}},
    description={Conduction band energy level}
}
\newglossaryentry{evalence}
{
    type=symbols,
    name={\ensuremath{E_{V}}},
    description={Valence band energy level}
}
\newglossaryentry{egap}
{
    type=symbols,
    name={\ensuremath{E_{G}}},
    description={Bandgap}
}
\newglossaryentry{temp}
{
    type=symbols,
    name={\ensuremath{T}},
    description={Temperature (\si{\kelvin})}
}
\newglossaryentry{rd}
{
    type=symbols,
    name={\ensuremath{R_{D}}},
    description={Resistance}
}
\newglossaryentry{vout}
{
    type=symbols,
    name={\ensuremath{V_{out}}},
    description={Output voltage}
}
\newglossaryentry{vin}
{
    type=symbols,
    name={\ensuremath{V_{in}}},
    description={Input voltage}
}

%%%%%%%%%%%%%%%%%%%
%%%%%%%%%%%%%%%% Acronym Only
\newglossaryentry{ndfeb}
{
    type=\acronymtype,
    name={NdFeB},
    description={Neodymium Iron Boron Rare Earth Permanent Magnet},
    first={Neodymium iron boron (NdFeB)}
}
\newglossaryentry{smco}
{
    type=\acronymtype,
    name={SmCo},
    description={Samarium Cobalt Rare Earth Permanent Magnet},
    first={Samarium cobalt (SmCo)}
}
\newglossaryentry{cnc}
{
    type=\acronymtype,
    name={CNC},
    description={Computer Numerically Controlled},
    first={Computer Numerically Controlled (CNC)}
}
\newglossaryentry{ptfe}
{
    type=\acronymtype,
    name={PTFE},
    description={Polytetraflouroethylene (PTFE), commonly referred to by the brand name Teflon, is a synthetic polymer with highly desirable electrical breakdown characteristics},
    first={Polytetraflouroethylene (PTFE)}
}

%%%%%%%%%%%%%%%%%%%
%%%%%%%%%%%%%%%% Glossary Only
\newglossaryentry{cpp}
{%
    name={C++},
    description={C++ is a programming language that can be used as an object oriented programming language, an imperative programming language, and still provide low-level memory control. Note: All C++ code used in this work is compiled under the C++11 standard}
}
\newglossaryentry{python}
{%
    name={Python},
    description={Python is a flexible scripting language that can be used as an imperative language, an object oriented language. Note: All python code used in this work is written and executed using Python3.6}
}
\newglossaryentry{ibsimu}
{%
    name={IBSIMU},
    description={IBSIMU is a \gls{cpp} library for ion optics, space-charge calculation, and plasma extraction. It is able to import 2-D and 3-D geometries as exported in .STL or .DXF file formats. It also allows for mathematical definition of geometries. Note: Version 1.0.6 of the IBSIMU library was used for this work \cite{acc:14}}
}
\newglossaryentry{ansys}
{
    name={ANSYS},
    description={ANSYS Multiphysics is a software suite for multiphysics modelling and simulation}
}

%%%%%%%%%%%%%%%%%%%
%%%%%%%%%%%%%%%% Glossary and Acronym
\newglossaryentry{uhvag}
{%
 name={UHV},
 description={Ultra High Vacuum (UHV) is a term which refers to vacuum pressures in the range of \SI{1e-3}{\torr} to \SI{1e-8}{\torr}\cite{acc:13}}
}
\newglossaryentry{uhva}
{%
 type=\acronymtype,
 name={UHV},
 description={Ultra High Vacuum},
 first={Ultra High Vacuum (UHV)\glsadd{uhvag}},
 see=[Glossary:]{uhvag}
}
\DTMsavedate{presentation}{2023-01-25}
\title{Hello \LaTeX}
\subtitle{An Introduction to the Typesetting Tool}
\author{Aaron English}
\date{\DTMusedate{presentation}}

\begin{document}
    \begin{frame}[noframenumbering, plain]
        \titlepage
    \end{frame}
        \note[itemize]{
          \item introduce myself
          \item name
          \item phd student
          \item research in fusion
          \item like open source
        }
    % \section*{Workshop Introduction}
        \begin{frame}
            \frametitle{Workshop Series}
            \begin{enumerate}[Workshop 1:]
                \addtolength{\itemindent}{4em}
                \item An introduction to \LaTeX
                \item More Powerful \LaTeX~Features
                \item Modularity and \LaTeX~in Different Environments
                \item Reproducibility and Experimental \LaTeX~Usage
            \end{enumerate}
        \end{frame}
        \note[itemize]{
          \item Introduce workshop series
          \item broken up into 4
          \item will explore increasingly sophisticated use
          \item describe sessions
        }
        \begin{frame}
            \frametitle{Workshop Layout}
            \begin{enumerate}
                \item Introduction to and Motivation for Learning \LaTeX
                \item Pizza Break
                \item Live coding walkthrough
                \item Let's Go!
            \end{enumerate}
        \end{frame}
        \note[itemize]{
          \item describe this session
        }
    \section{Motivation}\label{sct:motiv}
        \begin{frame}
            \frametitle{Setting the Stage}
            \begin{itemize}
                \item Bad documents cost the reader energy
                \item Good documents cost the writer energy
                \item Eliminate (or mitigate) the tradeoff with \LaTeX
            \end{itemize}
        \end{frame}
        \note[itemize]{
          \item bad doc
          \begin{itemize}
            \item inconsistent notation
            \item inconsistent font
            \item out of date references to material
            \item inaccurate values
            \item inconsistent units
          \end{itemize}
          \item good doc cost examples
          \item revisiting your work carries double cost
          \item assumes word processors are the only choice
        }
        \begin{frame}
            \frametitle{Introducing \LaTeX}
            \begin{itemize}
                \item A typesetting tool - Not WYSIWYG
                \item Decouples content from formatting
                \item The de facto standard in research and academia
                \item One of the most sophisticated typesetting tools available
            \end{itemize}
        \end{frame}
        \note[itemize]{
          \item A different approach to word processors
          \item latex takes that and compiles it
          \item originally developed in the 80's repeatedly shown to be the most powerful
        }
        \begin{frame}
            \frametitle{Notable Features of \LaTeX}
            \begin{itemize}
                \item Consistent high quality documents
                \item Reduce your cognitive load - when reading and writing
                \item Automation
                \item Modularize your work
                \item Open source
                \item Extensible
            \end{itemize}
        \end{frame}
        \note[itemize]{
          \item programmatic, enforces consistency
          \item you define what you want, and how you want it
          \item latex takes that and compiles it
          \item not constrained to metaphors of real world documents
          \item use the computer to take the burden of minutiae
          \item highly modular, reuse your work
          \item open source, extensible
          \item explore examples
        }
    \section{Basics of Preparing a \LaTeX~Document}
        \begin{frame}
            \frametitle{The Environment - Overleaf}
            \begin{itemize}
                \item Cloud based - consistent build environments
                \item Collaborative editing capabilities
                \item Good place to start
                \begin{figure}
                    \begin{centering}
                        \includegraphics[width=\linewidth]{Images/overleafEnv.png}
                        \caption{The Overleaf editing environment}
                        \label{fig:overleafEnv}
                    \end{centering}
                \end{figure}
            \end{itemize}
        \end{frame}
        \begin{frame}[fragile]
            \frametitle{Commands}
            \begin{minted}{tex}
                \command[<some options>]{<arguments>}
            \end{minted}
            \begin{itemize}
                \item Differentiate \LaTeX~instructions from text
                \item optional arguments passed in \mintinline{tex}{[]}
                \item Required arguments passed in \mintinline{tex}{{}}
            \end{itemize}
        \end{frame}
        \begin{frame}[fragile]
            \frametitle{The \mintinline{tex}{.tex} File}
                \begin{columns}[onlytextwidth]%
                    \begin{column}{0.4\textwidth}%
                        \begin{itemize}
                            \item<1> A plain-text file
                            \item<1> Two main parts
                            \item<2|visible@2-> Preamble - Setup
                            \item<3|visible@3> Body - Content
                        \end{itemize}
                    \end{column}
                    \begin{column}{0.6\textwidth}%
                        \begin{onlyenv}<1>
                            \begin{minted}{tex}
                                \documentclass[hidelinks, 12pt]{article}
                                \usepackage{geometry}
                                \usepackage{hyperref}
                                \usepackage[tbtags]{amsmath}
                                \usepackage{amsfonts}
                                \usepackage{amssymb}
                                \usepackage[utf8]{inputenc}
                                \usepackage[T1]{fontenc}
                                \begin{document}
                                    <your document code>
                                \end{document}
                            \end{minted}
                        \end{onlyenv}
                        \begin{onlyenv}<2>
                            \begin{minted}[highlightlines={1-8}]{tex}
                                \documentclass[hidelinks, 12pt]{article}
                                \usepackage{geometry}
                                \usepackage{hyperref}
                                \usepackage[tbtags]{amsmath}
                                \usepackage{amsfonts}
                                \usepackage{amssymb}
                                \usepackage[utf8]{inputenc}
                                \usepackage[T1]{fontenc}
                                \begin{document}
                                    <your document code>
                                \end{document}
                            \end{minted}
                        \end{onlyenv}
                        \begin{onlyenv}<3>
                            \begin{minted}[highlightlines={9-11}]{tex}
                                \documentclass[hidelinks, 12pt]{article}
                                \usepackage{geometry}
                                \usepackage{hyperref}
                                \usepackage[tbtags]{amsmath}
                                \usepackage{amsfonts}
                                \usepackage{amssymb}
                                \usepackage[utf8]{inputenc}
                                \usepackage[T1]{fontenc}
                                \begin{document}
                                    <your document code>
                                \end{document}
                            \end{minted}
                        \end{onlyenv}
                    \end{column}
                \end{columns}
        \end{frame}
        \begin{frame}[fragile]
            \frametitle{Preamble Elements}
                \begin{columns}[onlytextwidth]%
                    \begin{column}{0.4\textwidth}%
                        \begin{itemize}
                            \item<1|visible@1-> Document class
                            \item<2|visible@2-> Packages imports
                            \item<3|visible@3-> Additional document setup
                        \end{itemize}
                    \end{column}
                    \begin{column}{0.6\textwidth}%
                        \begin{onlyenv}<1>
                            \begin{minted}[highlightlines={1}]{tex}
                                \documentclass[hidelinks, 12pt]{article}
                                \usepackage{geometry}
                                \usepackage{hyperref}
                                \usepackage[tbtags]{amsmath}
                                \usepackage{amsfonts}
                                \usepackage{amssymb}
                                \usepackage[utf8]{inputenc}
                                \usepackage[T1]{fontenc}
                                ...
                                \title{My First Document}
                                \date{\today}
                                \author{Aaron English}
                            \end{minted}
                        \end{onlyenv}
                        \begin{onlyenv}<2>
                            \begin{minted}[highlightlines={2-8}]{tex}
                                \documentclass[hidelinks, 12pt]{article}
                                \usepackage{geometry}
                                \usepackage{hyperref}
                                \usepackage[tbtags]{amsmath}
                                \usepackage{amsfonts}
                                \usepackage{amssymb}
                                \usepackage[utf8]{inputenc}
                                \usepackage[T1]{fontenc}
                                ...
                                \title{My First Document}
                                \date{\today}
                                \author{Aaron English}
                            \end{minted}
                        \end{onlyenv}
                        \begin{onlyenv}<3>
                            \begin{minted}[highlightlines={10-12}]{tex}
                                \documentclass[hidelinks, 12pt]{article}
                                \usepackage{geometry}
                                \usepackage{hyperref}
                                \usepackage[tbtags]{amsmath}
                                \usepackage{amsfonts}
                                \usepackage{amssymb}
                                \usepackage[utf8]{inputenc}
                                \usepackage[T1]{fontenc}
                                ...
                                \title{My First Document}
                                \date{\today}
                                \author{Aaron English}
                            \end{minted}
                        \end{onlyenv}
                    \end{column}
                \end{columns}
        \end{frame}
        \begin{frame}[fragile]
            \frametitle{Body Elements}
                \begin{columns}[onlytextwidth]%
                    \begin{column}{0.4\textwidth}%
                        \begin{itemize}
                            \item<1|visible@1-> Begin document
                            \item<2|visible@2-> Add title
                            \item<3|visible@3-> Add table of contents
                            \item<4|visible@4-> Sections
                            \item<4|visible@4-> ... and subsections
                            \item<5|visible@5-> Add text
                            \item<6|visible@6-> Environments
                                \begin{itemize}
                                    \item Equations
                                    \item Figures
                                    \item Tables
                                    \item Code blocks
                                    \item and more
                                \end{itemize}
                        \end{itemize}
                    \end{column}
                    \begin{column}{0.6\textwidth}%
                        \begin{onlyenv}<1>
                            \begin{minted}[highlightlines={1-3}]{tex}
                                \begin{document}

                                \end{document}
                            \end{minted}
                        \end{onlyenv}
                        \begin{onlyenv}<2>
                            \begin{minted}[highlightlines={2}]{tex}
                                \begin{document}
                                  \maketitle
                                \end{document}
                            \end{minted}
                        \end{onlyenv}
                        \begin{onlyenv}<3>
                            \begin{minted}[highlightlines={3}]{tex}
                                \begin{document}
                                  \maketitle
                                  \tableofcontents
                                \end{document}
                            \end{minted}
                        \end{onlyenv}
                        \begin{onlyenv}<4>
                            \begin{minted}[highlightlines={4-5}]{tex}
                                \begin{document}
                                  \maketitle
                                  \tableofcontents
                                  \section{Introduction}
                                    \subsection{Sub-Intro}
                                \end{document}
                            \end{minted}
                        \end{onlyenv}
                        \begin{onlyenv}<5>
                            \begin{minted}[highlightlines={6}]{tex}
                                \begin{document}
                                  \maketitle
                                  \tableofcontents
                                  \section{Introduction}
                                    \subsection{Sub-Intro}
                                    Some actual text content
                                \end{document}
                            \end{minted}
                        \end{onlyenv}
                        \begin{onlyenv}<6>
                            \begin{minted}[highlightlines={7-10}]{tex}
                                \begin{document}
                                  \maketitle
                                  \tableofcontents
                                  \section{Introduction}
                                    \subsection{Sub-Intro}
                                    Some actual text content
                                    \begin{equation}
                                      a^{2} = b^{2} + c^{2}
                                      \label{eqt:pytha}
                                    \end{equation}
                                \end{document}
                            \end{minted}
                        \end{onlyenv}
                        \begin{onlyenv}<7>
                            \begin{minted}{tex}
                                \begin{document}
                                  \maketitle
                                  \tableofcontents
                                  \section{Introduction}
                                    \subsection{Sub-Intro}
                                    Some actual text content
                                    \begin{equation}
                                      a^{2} = b^{2} + c^{2}
                                      \label{eqt:pytha}
                                    \end{equation}
                                \end{document}
                            \end{minted}
                        \end{onlyenv}
                    \end{column}
                \end{columns}
        \end{frame}
    \section{High Quality Results}
        \begin{frame}[fragile]
            \frametitle{Equations}
            \begin{itemize}
                \item<only@1> Equations like Equation \ref{eqt:quadratic}
                    % \begin{minted}{tex}
                    % \begin{equation}
                    % 0 = a_{2}x^{2} + a_{1}x + a_{0}
                    % \label{eqt:quadratic}
                    % \end{equation}
                    % \end{minted}
                    \begin{equation}
                    0 = a_{2}x^{2} + a_{1}x + a_{0}
                    \label{eqt:quadratic}
                    \end{equation}
                \item<only@2> Or a fancier Equations like the Fourier transform
                    % \begin{minted}{tex}
                    % \begin{equation}
                    %   \hat{f}\left(\xi\right) = \int_{-\infty}^{\infty} f\left(\xi\right) e^{2\pi ix\xi}\mathrm{d}\xi
                    %   \label{eqt:fourTrans}
                    % \end{equation}
                    % \end{minted}
                    \begin{equation}
                    \hat{f}\left(\xi\right) = \int_{-\infty}^{\infty}f\left(\xi\right)e^{2\pi ix\xi}\text{d}\xi
                    \label{eqt:fourTrans}
                    \end{equation}
                \item<only@3> Figures and captions
                    % \begin{columns}[onlytextwidth]%
                    %     \begin{column}{0.5\textwidth}%
                    %         \begin{minted}{tex}
                    %         \begin{figure}
                    %           \begin{centering}
                    %             \includegraphics [width=0.8\textwidth] {Images/rootLocus.png}
                    %             \caption{A test figure}
                    %             \label{fig:test}
                    %           \end{centering}
                    %         \end{figure}
                    %         \end{minted}
                    %     \end{column}
                    %     \begin{column}{0.5\textwidth}%
                            \begin{figure}
                                \begin{centering}
                                    \includegraphics[width=0.8\textwidth]{Images/rootLocus.png}
                                    \caption{A test figure}
                                    \label{fig:test}
                                \end{centering}
                            \end{figure}
                    %     \end{column}
                    % \end{columns}
                \item<only@4> Code blocks from external files
                    \inputminted{python}{./demo.py}
                \item<only@5> Tables like the following
                %     \begin{minted}{tex}
                %       \begin{table}[htbp]
                %         \caption{Table of specific energies of different fuels}
                %         \label{tab:specificEnergies}
                %         \begin{tabular}{llr}
                %           \toprule
                %           \textbf{Reaction Kind} & \textbf{Fuel} & \textbf{\si{\joule\per\gram}}\\
                %           \midrule
                %           Nuclear Fusion & Deuterium & 5.71e11\\
                %           Nuclear Fission & Uranium & 8.06e10\\
                %           Oxidation & Methane (\ch{CH4}) & 5.56e4\\
                %           Electrochemical & Lithium Ion Battery & 6.18e2\\
                %           \bottomrule
                %         \end{tabular}
                %       \end{table}
                %     \end{minted}
                % \item<only@6> Tables like the following
                    \begin{table}[htbp]
                        \caption{Table of specific energies of different fuels}
                        \label{tab:specificEnergies}
                        \begin{tabular}{llr}
                            \toprule
                            \textbf{Reaction Kind} & \textbf{Fuel} & \textbf{\si{\joule\per\gram}}\\
                            \midrule
                            Nuclear Fusion & Deuterium & 5.71e11\\
                            Nuclear Fission & Uranium & 8.06e10\\
                            Oxidation & Methane (\ch{CH4}) & 5.56e4\\
                            Electrochemical & Lithium Ion Battery & 6.18e2\\
                            \bottomrule
                        \end{tabular}
                    \end{table}
            \end{itemize}
        \end{frame}
        \begin{frame}
            \frametitle{Automation, and Consistency}
            \begin{itemize}
                \item bibliography
                    Citations are a breeze. \cite{acc:4}
                    And the counting and formatting of the citations is handled based on the style I select! \cite{acc:6}
                \item equation ref and numbering
                    I can refer to any equation (like Equation \ref{eqt:quadratic}) or any figure (Figure \ref{fig:test}) or table (Table \ref{tab:specificEnergies}) and the order will always be correct, every time I compile the document
            \end{itemize}
        \end{frame}
        \begin{frame}[fragile]
            \frametitle{Modularity}
            \begin{itemize}
                \item Large blocks of code are highly recyclable
                    \begin{itemize}
                        \item Preambles
                        \item Environment setups
                        \item \mintinline{tex}{.bib} Bibliographies
                    \end{itemize}
            \end{itemize}
        \end{frame}
        \begin{frame}
            \frametitle{Open Source}
            \begin{itemize}
                \item Easily accessed
              \item Large active community
                \item
            \end{itemize}
          \end{frame}
    \section{Support and Resources}
        \begin{frame}
            \frametitle{Bibliography Generation}
            \begin{itemize}
                \item acronyms and glossaries
                \item bibliography
                \item equation ref and numbering
            \end{itemize}
        \end{frame}
        \begin{frame}
            \frametitle{This Code and Other Examples}
            \begin{itemize}
                \item acronyms and glossaries
                \item bibliography
                \item equation ref and numbering
            \end{itemize}
        \end{frame}
        \begin{frame}
            \frametitle{Troubleshooting and Questions}
            \begin{itemize}
                \item acronyms and glossaries
                \item bibliography
                \item equation ref and numbering
            \end{itemize}
        \end{frame}
    \section{Backmatter}
        \begin{frame}[allowframebreaks, noframenumbering, plain]
            \frametitle{Bibliography}
            \printbibliography
        \end{frame}
        \begin{frame}[allowframebreaks, noframenumbering, plain]
            \frametitle{Acronyms}
            \printglossary[type=\acronymtype]
        \end{frame}
        \begin{frame}[allowframebreaks, noframenumbering, plain]
            \frametitle{Glossary}
            \printglossary[type=main]
        \end{frame}
        \begin{frame}[allowframebreaks, noframenumbering, plain]
            \frametitle{Constants}
            \printglossary[type=constants, nonumberlist, nopostdot]
        \end{frame}
        \begin{frame}[allowframebreaks, noframenumbering, plain]
            \frametitle{Symbols}
            \printglossary[type=symbols]
        \end{frame}
\end{document}
