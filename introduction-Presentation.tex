% arara: lualatex: { shell: yes }
% arara: biber
% arara: lualatex: { shell: yes }
% arara: lualatex: { shell: yes }


% Demo report and presentation prepared by Aaron English (humdrumcomet on github)
\documentclass{beamer}
% Beamer format setup
\usetheme[progressbar=frametitle, numbering=fraction]{metropolis}
\setbeameroption{show notes on second screen}
\setbeamertemplate{note page}[plain]
\setbeamerfont{note page}{size=\Large}
\setbeamercovered{transparent, still covered={\opaqueness<1->{65}},again covered={\opaqueness<1->{65}}}

\usepackage{fontspec}
\usepackage{geometry} % useful for defining page geometries
\usepackage[style=ieee,backend=biber]{biblatex} % for handling bibliographies
\usepackage{float} % added control over
\usepackage{graphicx} % tools for inclusion of graphics
\usepackage{booktabs} % adding commands to improve the look of tables
\usepackage{csvsimple} % simplify table creation by importing .csv files directly
\usepackage{siunitx} % consistent notation and correct formatting of units
\DeclareSIUnit{\torr}{Torr} % Custom unit definition
\usepackage{xcolor} % to access the named colour LightGray
\definecolor{LightGray}{gray}{0.9}
\usepackage{minted} % inclusion of code blocks with syntax highlighting and
\setminted{linenos=flase, autogobble, fontsize=\scriptsize, breaklines=true, bgcolor=LightGray} % set global options for minted environments
\setmintedinline{bgcolor={}}
\usepackage{chemformula} % for writing chemical formulae
\usepackage[useregional]{datetime2}
\usepackage[duration=30]{pdfpc}


% siunitx package setup
\sisetup{
load-configurations = abbreviations,
detect-family = true,
detect-weight = true,
per-mode = reciprocal,
input-digits = { 0123456789\pi\dots },
input-comparators = { <=>\approx\ge\geq\gg\le\leq\ll\sim\gtrsim\lesssim },
table-figures-integer = 3,
table-figures-decimal = 2,
table-auto-round,
}%

% Bibliography .bib file locations should maybe be local?
\addbibresource[location=local]{./accelerators.bib}
\DTMsavedate{presentation}{2023-01-25}

\title{Hello \LaTeX}
\subtitle{An Introduction to the Typesetting Tool}
\author{Aaron English}
\date{\DTMusedate{presentation}}

\begin{document}
    \begin{frame}[noframenumbering, plain]
        \titlepage
    \end{frame}
        \note[itemize]{
          \item introduce myself
          \item name
          \item phd student
          \item research in fusion
          \item like open source
        }
    % \section*{Workshop Introduction}
        \begin{frame}
            \frametitle{Workshop Series}
            \begin{enumerate}[Workshop 1:]
                \addtolength{\itemindent}{4em}
                \item An introduction to \LaTeX
                \item More Powerful \LaTeX~Features
                \item Modularity and \LaTeX~in Different Environments
                \item Reproducibility and Experimental \LaTeX~Usage
            \end{enumerate}
        \end{frame}
        \note[itemize]{
          \item Introduce workshop series
          \item broken up into 4
          \item will explore increasingly sophisticated use
          \item describe sessions
        }
        \begin{frame}
            \frametitle{Workshop Layout}
            \begin{enumerate}
                \item Introduction to and Motivation for Learning \LaTeX
                \item Pizza Break
                \item Live coding walkthrough
                \item Let's Go!
            \end{enumerate}
        \end{frame}
        \note[itemize]{
          \item describe this session
        }
    \section{Motivation}\label{sct:motiv}
        \begin{frame}
            \frametitle{Setting the Stage}
            \begin{itemize}
                \item Bad documents cost the reader energy
                \item Good documents cost the writer energy
                \item Eliminate (or mitigate) the tradeoff with \LaTeX
            \end{itemize}
        \end{frame}
        \note[itemize]{
          \item bad doc
          \begin{itemize}
            \item inconsistent notation
            \item inconsistent font
            \item out of date references to material
            \item inaccurate values
            \item inconsistent units
          \end{itemize}
          \item good doc cost examples
          \item revisiting your work carries double cost
          \item assumes word processors are the only choice
        }
        \begin{frame}
            \frametitle{Introducing \LaTeX}
            \begin{itemize}
                \item A typesetting tool - Not WYSIWYG
                \item Decouples content from formatting
                \item The de facto standard in research and academia
                \item One of the most sophisticated typesetting tools available
            \end{itemize}
        \end{frame}
        \note[itemize]{
          \item A different approach to word processors
          \item latex takes that and compiles it
          \item originally developed in the 80's repeatedly shown to be the most powerful
        }
        \begin{frame}
            \frametitle{Notable Features of \LaTeX}
            \begin{itemize}
                \item Consistent high quality documents
                \item Reduce your cognitive load - when reading and writing
                \item Automation
                \item Modularize your work
                \item Open source
                \item Extensible
            \end{itemize}
        \end{frame}
        \note[itemize]{
          \item programmatic, enforces consistency
          \item you define what you want, and how you want it
          \item latex takes that and compiles it
          \item not constrained to metaphors of real world documents
          \item use the computer to take the burden of minutiae
          \item highly modular, reuse your work
          \item open source, extensible
          \item explore examples
        }
    \section{Basics of Preparing a \LaTeX~Document}
        \begin{frame}
            \frametitle{The Environment - Overleaf}
            \begin{itemize}
                \item Cloud based - consistent build environments
                \item Collaborative editing capabilities
                \item Good place to start
                \begin{figure}
                    \begin{centering}
                        \includegraphics[width=\linewidth]{Images/overleafEnv.png}
                        \caption{The Overleaf editing environment}
                        \label{fig:overleafEnv}
                    \end{centering}
                \end{figure}
            \end{itemize}
        \end{frame}
        \begin{frame}[fragile]
            \frametitle{Commands}
            \begin{minted}{tex}
                \command[<some options>]{<arguments>}
            \end{minted}
            \begin{itemize}
                \item Differentiate \LaTeX~instructions from text
                \item Optional arguments passed in \mintinline{tex}{[]}
                \item Required arguments passed in \mintinline{tex}{{}}
            \end{itemize}
        \end{frame}
        \begin{frame}[fragile]
            \frametitle{The \mintinline{tex}{.tex} File}
                \begin{columns}[onlytextwidth]%
                    \begin{column}{0.4\textwidth}%
                        \begin{itemize}
                            \item<1> A plain-text file
                            \item<1> Two main parts
                            \item<2|visible@2-> Preamble - Setup
                            \item<3|visible@3> Body - Content
                        \end{itemize}
                    \end{column}
                    \begin{column}{0.6\textwidth}%
                        \begin{onlyenv}<1>
                            \begin{minted}{tex}
                                \documentclass[hidelinks, 12pt]{article}
                                \usepackage{geometry}
                                \usepackage{hyperref}
                                \usepackage[tbtags]{amsmath}
                                \usepackage{amsfonts}
                                \usepackage{amssymb}
                                \usepackage[utf8]{inputenc}
                                \usepackage[T1]{fontenc}
                                \begin{document}
                                    <your document code>
                                \end{document}
                            \end{minted}
                        \end{onlyenv}
                        \begin{onlyenv}<2>
                            \begin{minted}[highlightlines={1-8}]{tex}
                                \documentclass[hidelinks, 12pt]{article}
                                \usepackage{geometry}
                                \usepackage{hyperref}
                                \usepackage[tbtags]{amsmath}
                                \usepackage{amsfonts}
                                \usepackage{amssymb}
                                \usepackage[utf8]{inputenc}
                                \usepackage[T1]{fontenc}
                                \begin{document}
                                    <your document code>
                                \end{document}
                            \end{minted}
                        \end{onlyenv}
                        \begin{onlyenv}<3>
                            \begin{minted}[highlightlines={9-11}]{tex}
                                \documentclass[hidelinks, 12pt]{article}
                                \usepackage{geometry}
                                \usepackage{hyperref}
                                \usepackage[tbtags]{amsmath}
                                \usepackage{amsfonts}
                                \usepackage{amssymb}
                                \usepackage[utf8]{inputenc}
                                \usepackage[T1]{fontenc}
                                \begin{document}
                                    <your document code>
                                \end{document}
                            \end{minted}
                        \end{onlyenv}
                    \end{column}
                \end{columns}
        \end{frame}
        \begin{frame}[fragile]
            \frametitle{Preamble Elements}
                \begin{columns}[onlytextwidth]%
                    \begin{column}{0.4\textwidth}%
                        \begin{itemize}
                            \item<1|visible@1-> Document class
                            \item<2|visible@2-> Packages imports
                            \item<3|visible@3-> Additional document setup
                        \end{itemize}
                    \end{column}
                    \begin{column}{0.6\textwidth}%
                        \begin{onlyenv}<1>
                            \begin{minted}[highlightlines={1}]{tex}
                                \documentclass[hidelinks, 12pt]{article}
                                \usepackage{geometry}
                                \usepackage{hyperref}
                                \usepackage[tbtags]{amsmath}
                                \usepackage{amsfonts}
                                \usepackage{amssymb}
                                \usepackage[utf8]{inputenc}
                                \usepackage[T1]{fontenc}
                                ...
                                \title{My First Document}
                                \date{\today}
                                \author{Aaron English}
                            \end{minted}
                        \end{onlyenv}
                        \begin{onlyenv}<2>
                            \begin{minted}[highlightlines={2-8}]{tex}
                                \documentclass[hidelinks, 12pt]{article}
                                \usepackage{geometry}
                                \usepackage{hyperref}
                                \usepackage[tbtags]{amsmath}
                                \usepackage{amsfonts}
                                \usepackage{amssymb}
                                \usepackage[utf8]{inputenc}
                                \usepackage[T1]{fontenc}
                                ...
                                \title{My First Document}
                                \date{\today}
                                \author{Aaron English}
                            \end{minted}
                        \end{onlyenv}
                        \begin{onlyenv}<3>
                            \begin{minted}[highlightlines={10-12}]{tex}
                                \documentclass[hidelinks, 12pt]{article}
                                \usepackage{geometry}
                                \usepackage{hyperref}
                                \usepackage[tbtags]{amsmath}
                                \usepackage{amsfonts}
                                \usepackage{amssymb}
                                \usepackage[utf8]{inputenc}
                                \usepackage[T1]{fontenc}
                                ...
                                \title{My First Document}
                                \date{\today}
                                \author{Aaron English}
                            \end{minted}
                        \end{onlyenv}
                    \end{column}
                \end{columns}
        \end{frame}
        \begin{frame}[fragile]
            \frametitle{Body Elements}
                \begin{columns}[onlytextwidth]%
                    \begin{column}{0.4\textwidth}%
                        \begin{itemize}
                            \item<1|visible@1-> Begin document
                            \item<2|visible@2-> Add title
                            \item<3|visible@3-> Add table of contents
                            \item<4|visible@4-> Sections
                            \item<4|visible@4-> ... and subsections
                            \item<5|visible@5-> Add text
                            \item<6|visible@6-> Environments
                                \begin{itemize}
                                    \item Equations
                                    \item Figures
                                    \item Tables
                                    \item Code blocks
                                    \item and more
                                \end{itemize}
                        \end{itemize}
                    \end{column}
                    \begin{column}{0.6\textwidth}%
                        \begin{onlyenv}<1>
                            \begin{minted}[highlightlines={1-3}]{tex}
                                \begin{document}

                                \end{document}
                            \end{minted}
                        \end{onlyenv}
                        \begin{onlyenv}<2>
                            \begin{minted}[highlightlines={2}]{tex}
                                \begin{document}
                                  \maketitle
                                \end{document}
                            \end{minted}
                        \end{onlyenv}
                        \begin{onlyenv}<3>
                            \begin{minted}[highlightlines={3}]{tex}
                                \begin{document}
                                  \maketitle
                                  \tableofcontents
                                \end{document}
                            \end{minted}
                        \end{onlyenv}
                        \begin{onlyenv}<4>
                            \begin{minted}[highlightlines={4-5}]{tex}
                                \begin{document}
                                  \maketitle
                                  \tableofcontents
                                  \section{Introduction}
                                    \subsection{Sub-Intro}
                                \end{document}
                            \end{minted}
                        \end{onlyenv}
                        \begin{onlyenv}<5>
                            \begin{minted}[highlightlines={6}]{tex}
                                \begin{document}
                                  \maketitle
                                  \tableofcontents
                                  \section{Introduction}
                                    \subsection{Sub-Intro}
                                    Some actual text content
                                \end{document}
                            \end{minted}
                        \end{onlyenv}
                        \begin{onlyenv}<6>
                            \begin{minted}[highlightlines={7-10}]{tex}
                                \begin{document}
                                  \maketitle
                                  \tableofcontents
                                  \section{Introduction}
                                    \subsection{Sub-Intro}
                                    Some actual text content
                                    \begin{equation}
                                      a^{2} = b^{2} + c^{2}
                                      \label{eqt:pytha}
                                    \end{equation}
                                \end{document}
                            \end{minted}
                        \end{onlyenv}
                        \begin{onlyenv}<7>
                            \begin{minted}{tex}
                                \begin{document}
                                  \maketitle
                                  \tableofcontents
                                  \section{Introduction}
                                    \subsection{Sub-Intro}
                                    Some actual text content
                                    \begin{equation}
                                      a^{2} = b^{2} + c^{2}
                                      \label{eqt:pytha}
                                    \end{equation}
                                \end{document}
                            \end{minted}
                        \end{onlyenv}
                    \end{column}
                \end{columns}
        \end{frame}
\end{document}
